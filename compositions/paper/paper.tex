\documentclass[review=true, 10pt, sigplan]{acmart}
\usepackage[utf8]{inputenc}
\usepackage[T1]{fontenc}

\newcommand{\packageGraphicx}{\usepackage{graphicx}}
\newcommand{\packageHyperref}{\usepackage{hyperref}}
\newcommand{\renewrmdefault}{\renewcommand{\rmdefault}{ptm}}
\newcommand{\packageRelsize}{\usepackage{relsize}}
% amsmath is required for the combination of {mathabx,
% wasysym, newtxmath} to work. Otherwise, newtxmath
% would load amsmath *after* mathabx and wasysym,
% causing command redefinition issues.
\newcommand{\packageAmsmath}{\usepackage{amsmath}}
\newcommand{\packageMathabx}{\usepackage{mathabx}}
% Avoid conflicts between "mathabx" and "wasysym",
% and between "wasysym" integrals and "amsmath" integrals (iint).
\newcommand{\packageWasysym}{
  \let\leftmoon\relax \let\rightmoon\relax \let\fullmoon\relax \let\newmoon\relax \let\diameter\relax
  \usepackage[nointegrals]{wasysym}}
% Both newtxmath and mathabx define the \widering command.
% The only reason we choose the newtxmath version is that
% acmart.cls is also using the one from newtxmath.
\newcommand{\packageTxfonts}{
  \let\widering\relax
  \let\oldwidebar\widebar
  \let\widebar\relax
  \usepackage{newtxmath}
  % if newtxmath is before version 1.7.0,
  % then we are still going to use widebar from mathabx
  \ifx\widebar\relax
    \let\widebar\oldwidebar
  \fi
}
\newcommand{\packageTextcomp}{\usepackage{textcomp}}
\newcommand{\packageFramed}{\usepackage{framed}}
\newcommand{\packageHyphenat}{\usepackage[htt]{hyphenat}}
\newcommand{\packageColor}{\usepackage[usenames,dvipsnames]{color}}
\newcommand{\doHypersetup}{\hypersetup{bookmarks=true,bookmarksopen=true,bookmarksnumbered=true}}
\newcommand{\packageTocstyle}{}
\newcommand{\packageCJK}{\IfFileExists{CJK.sty}{\usepackage{CJK}}{}}
%%%%%%%%%%%%%%%%%%%%%%%%%%%%%%%%%%%%%%%%%%%%%%%%%%%%%%%%%%%%%%%%%%%%%%%%%%%%%%%%
% BEGIN acmart-load.tex
% Avoid package option conflict
\renewcommand\packageColor\relax
\renewcommand\packageTocstyle\relax
\renewcommand\packageMathabx{\ifx\bigtimes\undefined \usepackage{mathabx} \else \relax \fi}
% Both 'mathabx' and 'newtxmath' (required by the 'acmart' class) define a '\bigtimes' command. 
\renewcommand\packageTxfonts\relax
\let\Footnote\undefined
\let\captionwidth\undefined
\renewcommand{\renewrmdefault}{}
% END acmart-load.tex
%%%%%%%%%%%%%%%%%%%%%%%%%%%%%%%%%%%%%%%%%%%%%%%%%%%%%%%%%%%%%%%%%%%%%%%%%%%%%%%%
% This is the default style configuration for Scribble-generated Latex

\packageGraphicx
\packageHyperref
\renewrmdefault
\packageRelsize
\packageAmsmath
\packageMathabx
\packageWasysym
\packageTxfonts
\packageTextcomp
\packageFramed
\packageHyphenat
\packageColor
\doHypersetup
\packageTocstyle
\packageCJK


%%%%%%%%%%%%%%%%%%%%%%%%%%%%%%%%%%%%%%%%%%%%%%%%%%%%%%%%%%%%%%%%%%%%%%%%%%%%%%%%
% Configuration that is especially meant to be overridden:

% Inserted before every ``chapter'', useful for starting each one on a new page:
\newcommand{\sectionNewpage}{}
% Inserted before every book ``part''
\newcommand{\partNewpage}{\sectionNewpage}

% Hooks for actions within the `document' environment:
\newcommand{\preDoc}{}
\newcommand{\postDoc}{}

% Generated by `secref'; first arg is section number, second is section title:
\newcommand{\BookRef}[2]{\emph{#2}}
\newcommand{\ChapRef}[2]{\SecRef{#1}{#2}}
\newcommand{\SecRef}[2]{section~#1}
\newcommand{\PartRef}[2]{part~#1}
% Generated by `Secref':
\newcommand{\BookRefUC}[2]{\BookRef{#1}{#2}}
\newcommand{\ChapRefUC}[2]{\SecRefUC{#1}{#2}}
\newcommand{\SecRefUC}[2]{Section~#1}
\newcommand{\PartRefUC}[2]{Part~#1}

% Variants of the above with a label for an internal reference:
\newcommand{\BookRefLocal}[3]{\hyperref[#1]{\BookRef{#2}{#3}}}
\newcommand{\ChapRefLocal}[3]{\hyperref[#1]{\ChapRef{#2}{#3}}}
\newcommand{\SecRefLocal}[3]{\hyperref[#1]{\SecRef{#2}{#3}}}
\newcommand{\PartRefLocal}[3]{\hyperref[#1]{\PartRef{#2}{#3}}}
\newcommand{\BookRefLocalUC}[3]{\hyperref[#1]{\BookRefUC{#2}{#3}}}
\newcommand{\ChapRefLocalUC}[3]{\hyperref[#1]{\ChapRefUC{#2}{#3}}}
\newcommand{\SecRefLocalUC}[3]{\hyperref[#1]{\SecRefUC{#2}{#3}}}
\newcommand{\PartRefLocalUC}[3]{\hyperref[#1]{\PartRefUC{#2}{#3}}}

% Variants of the above with a section number is empty (i.e., UnNumbered):
\newcommand{\BookRefUN}[1]{\BookRef{}{#1}}
\newcommand{\ChapRefUN}[1]{\SecRefUN{#1}}
\newcommand{\SecRefUN}[1]{``#1''}
\newcommand{\PartRefUN}[1]{\SecRefUN{#1}}
\newcommand{\BookRefUCUN}[1]{\BookRefUN{#1}}
\newcommand{\ChapRefUCUN}[1]{\ChapRefUN{#1}}
\newcommand{\SecRefUCUN}[1]{\SecRefUN{#1}}
\newcommand{\PartRefUCUN}[1]{\PartRefUN{#1}}

\newcommand{\BookRefLocalUN}[2]{\hyperref[#1]{\BookRefUN{#2}}}
\newcommand{\ChapRefLocalUN}[2]{\SecRefLocalUN{#1}{#2}}
\newcommand{\SecRefLocalUN}[2]{\hyperref[#1]{\SecRefUN{#2}}}
\newcommand{\PartRefLocalUN}[2]{\SecRefLocalUN{#1}{#2}}
\newcommand{\BookRefLocalUCUN}[2]{\BookRefLocalUN{#1}{#2}}
\newcommand{\ChapRefLocalUCUN}[2]{\ChapRefLocalUN{#1}{#2}}
\newcommand{\SecRefLocalUCUN}[2]{\SecRefLocalUN{#1}{#2}}
\newcommand{\PartRefLocalUCUN}[2]{\PartRefLocalUN{#1}{#2}}

\newcommand{\SectionNumberLink}[2]{\hyperref[#1]{#2}}

% Enabled with a 'enable-index-merge part style property. This default
% implementation isn't good enough, because the argument is a
% comma-separated sequence of labels:
\newcommand{\Smanypageref}[1]{\pageref{#1}}

%%%%%%%%%%%%%%%%%%%%%%%%%%%%%%%%%%%%%%%%%%%%%%%%%%%%%%%%%%%%%%%%%%%%%%%%%%%%%%%%
% Fonts

% Font commands used by generated text:
\newcommand{\Scribtexttt}[1]{{\texttt{#1}}}
\newcommand{\textsub}[1]{$_{\hbox{\textsmaller{#1}}}$}
\newcommand{\textsuper}[1]{$^{\hbox{\textsmaller{#1}}}$}
\newcommand{\intextcolor}[2]{\textcolor{#1}{#2}}
\newcommand{\intextrgbcolor}[2]{\textcolor[rgb]{#1}{#2}}
\newcommand{\incolorbox}[2]{{\fboxrule=0pt\fboxsep=0pt\protect\colorbox{#1}{#2}}}
\newcommand{\inrgbcolorbox}[2]{{\fboxrule=0pt\fboxsep=0pt\protect\colorbox[rgb]{#1}{#2}}}
\newcommand{\plainlink}[1]{#1}
\newcommand{\techoutside}[1]{#1}
\newcommand{\techinside}[1]{#1}
\newcommand{\badlink}[1]{#1}
\newcommand{\indexlink}[1]{#1}
\newcommand{\noborder}[1]{#1}
\newcommand{\Smaller}[1]{\textsmaller{#1}}
\newcommand{\Larger}[1]{\textlarger{#1}}
\newcommand{\planetName}[1]{PLane\hspace{-0.1ex}T}
\newcommand{\slant}[1]{{\textsl{#1}}}

% Used for <, >, and | in tt mode. For some fonts and installations,
% there seems to be an encoding issue, so pick T1 explicitly:
\newcommand{\Stttextmore}{{\fontencoding{T1}\selectfont>}}
\newcommand{\Stttextless}{{\fontencoding{T1}\selectfont<}}
\newcommand{\Stttextbar}{{\fontencoding{T1}\selectfont|}}

%%%%%%%%%%%%%%%%%%%%%%%%%%%%%%%%%%%%%%%%%%%%%%%%%%%%%%%%%%%%%%%%%%%%%%%%%%%%%%%%
% Tables

% The `stabular' environment seems to be the lesser of evils among 
%  page-breaking table environments (and we've made a copy as ``pltstabular'
%  to make sure that it doesn't change).

\makeatletter
%%%%%%%%%%%%%%%%%%%%%%%%%%%%%%%%%%%%%%%%%%%%%%%%%%%%%%%%%%%%%%%%%%%%%%
\message{pltstabular is a modification of stabular}
%% A renamed version of:
%% stabular.sty
%% Copyright 1998 Sigitas Tolu\v sis
%% VTeX Ltd., Akademijos 4, Vilnius, Lithuania
%% e-mail sigitas@vtex.lt
%% http://www.vtex.lt/tex/download/macros/
%%
% This program can redistributed and/or modified under the terms
% of the LaTeX Project Public License Distributed from CTAN
% archives in directory macros/latex/base/lppl.txt; either
% version 1 of the License, or (at your option) any later version.
%
% PURPOSE:   Improve tabular environment.
%
% SHORT DESCRIPTION:
%
% Changed internal commands: \@mkpream, \@addamp, \@xhline
%
% Provides new commands in tabular (used after command \\):
% \emptyrow[#1] 
% -------------
%    Adds empty row, #1 - height of the row 
%
% \tabrow{#1}[#2] 
% ---------------
%    Adds row of natural height: #1\\[#2]
%
% Provides new environments: pltstabular and pltstabular* 
%                            --------     ---------
%            One more multi-page version of tabular
%
%
\def\empty@finalstrut#1{%
  \unskip\ifhmode\nobreak\fi\vrule\@width\z@\@height\z@\@depth\z@}
\def\no@strut{\global\setbox\@arstrutbox\hbox{%
    \vrule \@height\z@
           \@depth\z@
           \@width\z@}%
    \gdef\@endpbox{\empty@finalstrut\@arstrutbox\par\egroup\hfil}%
}%
\def\yes@strut{\global\setbox\@arstrutbox\hbox{%
    \vrule \@height\arraystretch \ht\strutbox
           \@depth\arraystretch \dp\strutbox
           \@width\z@}%
    \gdef\@endpbox{\@finalstrut\@arstrutbox\par\egroup\hfil}%
}%
\def\@mkpream#1{\@firstamptrue\@lastchclass6
  \let\@preamble\@empty\def\empty@preamble{\add@ins}%
  \let\protect\@unexpandable@protect
  \let\@sharp\relax\let\add@ins\relax
  \let\@startpbox\relax\let\@endpbox\relax
  \@expast{#1}%
  \expandafter\@tfor \expandafter
    \@nextchar \expandafter:\expandafter=\reserved@a\do
       {\@testpach\@nextchar
    \ifcase \@chclass \@classz \or \@classi \or \@classii \or \@classiii
      \or \@classiv \or\@classv \fi\@lastchclass\@chclass}%
  \ifcase \@lastchclass \@acol
      \or \or \@preamerr \@ne\or \@preamerr \tw@\or \or \@acol \fi}
\def\@addamp{%
  \if@firstamp
    \@firstampfalse
    \edef\empty@preamble{\add@ins}%
  \else
    \edef\@preamble{\@preamble &}%
    \edef\empty@preamble{\expandafter\noexpand\empty@preamble &\add@ins}%
  \fi}
\newif\iftw@hlines \tw@hlinesfalse
\def\@xhline{\ifx\reserved@a\hline
               \tw@hlinestrue
             \else\ifx\reserved@a\Hline
               \tw@hlinestrue
             \else
               \tw@hlinesfalse
             \fi\fi
      \iftw@hlines
        \aftergroup\do@after
      \fi
      \ifnum0=`{\fi}%
}
\def\do@after{\emptyrow[\the\doublerulesep]}
\def\emptyrow{\noalign\bgroup\@ifnextchar[\@emptyrow{\@emptyrow[\z@]}}
\def\@emptyrow[#1]{\no@strut\gdef\add@ins{\vrule \@height\z@ \@depth#1 \@width\z@}\egroup%
\empty@preamble\\
\noalign{\yes@strut\gdef\add@ins{\vrule \@height\z@ \@depth\z@ \@width\z@}}%
}
\def\tabrow#1{\noalign\bgroup\@ifnextchar[{\@tabrow{#1}}{\@tabrow{#1}[]}}
\def\@tabrow#1[#2]{\no@strut\egroup#1\ifx.#2.\\\else\\[#2]\fi\noalign{\yes@strut}}
%
\def\endpltstabular{\crcr\egroup\egroup \egroup}
\expandafter \let \csname endpltstabular*\endcsname = \endpltstabular
\def\pltstabular{\let\@halignto\@empty\@pltstabular}
\@namedef{pltstabular*}#1{\def\@halignto{to#1}\@pltstabular}
\def\@pltstabular{\leavevmode \bgroup \let\@acol\@tabacol
   \let\@classz\@tabclassz
   \let\@classiv\@tabclassiv \let\\\@tabularcr\@stabarray}
\def\@stabarray{\m@th\@ifnextchar[\@sarray{\@sarray[c]}}
\def\@sarray[#1]#2{%
  \bgroup
  \setbox\@arstrutbox\hbox{%
    \vrule \@height\arraystretch\ht\strutbox
           \@depth\arraystretch \dp\strutbox
           \@width\z@}%
  \@mkpream{#2}%
  \edef\@preamble{%
    \ialign \noexpand\@halignto
      \bgroup \@arstrut \@preamble \tabskip\z@skip \cr}%
  \let\@startpbox\@@startpbox \let\@endpbox\@@endpbox
  \let\tabularnewline\\%
%    \let\par\@empty
    \let\@sharp##%
    \set@typeset@protect
    \lineskip\z@skip\baselineskip\z@skip
    \@preamble}

%%%%%%%%%%%%%%%%%%%%%%%%%%%%%%%%%%%%%%%%%%%%%%%%%%%%%%%%%%%%%%%%%%%%%%
\makeatother

\newenvironment{bigtabular}{\begin{pltstabular}}{\end{pltstabular}}
% For the 'boxed table style:
\newcommand{\SBoxedLeft}{\textcolor[rgb]{0.6,0.6,1.0}{\vrule width 3pt\hspace{3pt}}}
% Formerly used to keep the horizontal line for a definition on the same page:
\newcommand{\SEndFirstHead}[0]{ \nopagebreak \\ }
% Corrects weirdness when a table is the first thing in
%  an itemization:
\newcommand{\bigtableinlinecorrect}[0]{~

\vspace{-\baselineskip}\vspace{\parskip}}
% Used to indent the table correctly in an itemization, since that's
%  one of the things stabular gets wrong:
\newlength{\stabLeft}
\newcommand{\bigtableleftpad}{\hspace{\stabLeft}}
\newcommand{\atItemizeStart}[0]{\addtolength{\stabLeft}{\labelsep}
                                \addtolength{\stabLeft}{\labelwidth}}


% For a single-column table in simple environments, it's better to
%  use the `list' environment instead of `stabular'.
\newenvironment{SingleColumn}{\begin{list}{}{\topsep=0pt\partopsep=0pt%
\listparindent=0pt\itemindent=0pt\labelwidth=0pt\leftmargin=0pt\rightmargin=0pt%
\itemsep=0pt\parsep=0pt}\item}{\end{list}}

%%%%%%%%%%%%%%%%%%%%%%%%%%%%%%%%%%%%%%%%%%%%%%%%%%%%%%%%%%%%%%%%%%%%%%%%%%%%%%%%
% Etc.

% ._ and .__
\newcommand{\Sendabbrev}[1]{#1\@}
\newcommand{\Sendsentence}[1]{\@#1}

% Default style for a nested flow:
\newenvironment{Subflow}{\begin{list}{}{\topsep=0pt\partopsep=0pt%
\listparindent=0pt\itemindent=0pt\labelwidth=0pt\leftmargin=0pt\rightmargin=0pt%
\itemsep=0pt}\item}{\end{list}}

% For the 'inset nested-flow style:
\newenvironment{SInsetFlow}{\begin{quote}}{\end{quote}}

% Indent a 'code-inset nested flow:
\newcommand{\SCodePreSkip}{\vskip\abovedisplayskip}
\newcommand{\SCodePostSkip}{\vskip\belowdisplayskip}
\newenvironment{SCodeFlow}{\SCodePreSkip\begin{list}{}{\topsep=0pt\partopsep=0pt%
\listparindent=0pt\itemindent=0pt\labelwidth=0pt\leftmargin=2ex\rightmargin=2ex%
\itemsep=0pt\parsep=0pt}\item}{\end{list}\SCodePostSkip}
\newcommand{\SCodeInsetBox}[1]{\setbox1=\hbox{\hbox{\hspace{2ex}#1\hspace{2ex}}}\vbox{\SCodePreSkip\vtop{\box1\SCodePostSkip}}}

% Inset a 'vertical-inset nested flow:
\newcommand{\SVInsetPreSkip}{\vskip\abovedisplayskip}
\newcommand{\SVInsetPostSkip}{\vskip\belowdisplayskip}
\newenvironment{SVInsetFlow}{\SVInsetPreSkip\begin{list}{}{\topsep=0pt\partopsep=0pt%
\listparindent=0pt\itemindent=0pt\labelwidth=0pt\leftmargin=0pt\rightmargin=0pt%
\itemsep=0pt\parsep=0pt}\item}{\end{list}\SVInsetPostSkip}
\newcommand{\SVInsetBox}[1]{\setbox1=\hbox{\hbox{#1}}\vbox{\SCodePreSkip\vtop{\box1\SCodePostSkip}}}

% The 'compact itemization style:
\newenvironment{compact}{\begin{itemize}}{\end{itemize}}
\newcommand{\compactItem}[1]{\item #1}

% The nested-flow style for `centerline':
\newenvironment{SCentered}{\begin{trivlist}\item \centering}{\end{trivlist}}

% The \refpara command corresponds to `margin-note'. The
% refcolumn and refcontent environments also wrap the note,
% because they simplify the CSS side.
\newcommand{\refpara}[1]{\normalmarginpar\marginpar{\raggedright \footnotesize #1}}
\newcommand{\refelem}[1]{\refpara{#1}}
\newenvironment{refcolumn}{}{}
\newenvironment{refcontent}{}{}

\newcommand{\refparaleft}[1]{\reversemarginpar\marginpar{\raggedright \footnotesize #1}}
\newcommand{\refelemleft}[1]{\refparaleft{#1}}
\newenvironment{refcolumnleft}{}{}

% Macros used by `title' and `author':
\newcommand{\titleAndVersionAndAuthors}[3]{\title{#1\\{\normalsize \SVersionBefore{}#2}}\author{#3}\maketitle}
\newcommand{\titleAndVersionAndEmptyAuthors}[3]{\title{#1\\{\normalsize \SVersionBefore{}#2}}#3\maketitle}
\newcommand{\titleAndEmptyVersionAndAuthors}[3]{\title{#1}\author{#3}\maketitle}
\newcommand{\titleAndEmptyVersionAndEmptyAuthors}[3]{\title{#1}\maketitle}
\newcommand{\titleAndVersionAndAuthorsAndShort}[4]{\title[#4]{#1\\{\normalsize \SVersionBefore{}#2}}\author{#3}\maketitle}
\newcommand{\titleAndVersionAndEmptyAuthorsAndShort}[4]{\title[#4]{#1\\{\normalsize \SVersionBefore{}#2}}#3\maketitle}
\newcommand{\titleAndEmptyVersionAndAuthorsAndShort}[4]{\title[#4]{#1}\author{#3}\maketitle}
\newcommand{\titleAndEmptyVersionAndEmptyAuthorsAndShort}[4]{\title[#4]{#1}\maketitle}
\newcommand{\SAuthor}[1]{#1}
\newcommand{\SAuthorSep}[1]{\qquad}
\newcommand{\SVersionBefore}[1]{Version }

% Useful for some styles, such as sigalternate:
\newcommand{\SNumberOfAuthors}[1]{}

\let\SOriginalthesubsection\thesubsection
\let\SOriginalthesubsubsection\thesubsubsection

% sections
\newcommand{\Spart}[2]{\part[#1]{#2}}
\newcommand{\Ssection}[2]{\section[#1]{#2}\let\thesubsection\SOriginalthesubsection}
\newcommand{\Ssubsection}[2]{\subsection[#1]{#2}\let\thesubsubsection\SOriginalthesubsubsection}
\newcommand{\Ssubsubsection}[2]{\subsubsection[#1]{#2}}
\newcommand{\Ssubsubsubsection}[2]{{\bf #2}}
\newcommand{\Ssubsubsubsubsection}[2]{\Ssubsubsubsection{#1}{#2}}

% "star" means unnumbered and not in ToC:
\newcommand{\Spartstar}[1]{\part*{#1}}
\newcommand{\Ssectionstar}[1]{\section*{#1}\renewcommand*\thesubsection{\arabic{subsection}}\setcounter{subsection}{0}}
\newcommand{\Ssubsectionstar}[1]{\subsection*{#1}\renewcommand*\thesubsubsection{\arabic{section}.\arabic{subsubsection}}\setcounter{subsubsection}{0}}
\newcommand{\Ssubsubsectionstar}[1]{\subsubsection*{#1}}
\newcommand{\Ssubsubsubsectionstar}[1]{{\bf #1}}
\newcommand{\Ssubsubsubsubsectionstar}[1]{\Ssubsubsubsectionstar{#1}}

% "starx" means unnumbered but in ToC:
\newcommand{\Spartstarx}[2]{\Spartstar{#2}\phantomsection\addcontentsline{toc}{part}{#1}}
\newcommand{\Ssectionstarx}[2]{\Ssectionstar{#2}\phantomsection\addcontentsline{toc}{section}{#1}}
\newcommand{\Ssubsectionstarx}[2]{\Ssubsectionstar{#2}\phantomsection\addcontentsline{toc}{subsection}{#1}}
\newcommand{\Ssubsubsectionstarx}[2]{\Ssubsubsectionstar{#2}\phantomsection\addcontentsline{toc}{subsubsection}{#1}}
\newcommand{\Ssubsubsubsectionstarx}[2]{\Ssubsubsubsectionstar{#2}}
\newcommand{\Ssubsubsubsubsectionstarx}[2]{\Ssubsubsubsubsectionstar{#2}}

% "grouper" is for the 'grouper style variant --- on subsections and lower,
%  because \Spart is used for grouper at the section level. Grouper implies
%  unnumbered.
\newcounter{GrouperTemp}
\newcommand{\Ssubsectiongrouper}[2]{\setcounter{GrouperTemp}{\value{subsection}}\Ssubsectionstarx{#1}{#2}\setcounter{subsection}{\value{GrouperTemp}}}
\newcommand{\Ssubsubsectiongrouper}[2]{\setcounter{GrouperTemp}{\value{subsubsection}}\Ssubsubsectionstarx{#1}{#2}\setcounter{subsubsection}{\value{GrouperTemp}}}
\newcommand{\Ssubsubsubsectiongrouper}[2]{\Ssubsubsubsectionstarx{#1}{#2}}
\newcommand{\Ssubsubsubsubsectiongrouper}[2]{\Ssubsubsubsubsectionstarx{#1}{#2}}

\newcommand{\Ssubsectiongrouperstar}[1]{\setcounter{GrouperTemp}{\value{subsection}}\Ssubsectionstar{#1}\setcounter{subsection}{\value{GrouperTemp}}}
\newcommand{\Ssubsubsectiongrouperstar}[1]{\setcounter{GrouperTemp}{\value{subsubsection}}\Ssubsubsectionstar{#1}\setcounter{subsubsection}{\value{GrouperTemp}}}
\newcommand{\Ssubsubsubsectiongrouperstar}[1]{\Ssubsubsubsectionstar{#1}}
\newcommand{\Ssubsubsubsubsectiongrouperstar}[1]{\Ssubsubsubsubsectionstar{#1}}

\newcommand{\Ssubsectiongrouperstarx}[2]{\setcounter{GrouperTemp}{\value{subsection}}\Ssubsectionstarx{#1}{#2}\setcounter{subsection}{\value{GrouperTemp}}}
\newcommand{\Ssubsubsectiongrouperstarx}[2]{\setcounter{GrouperTemp}{\value{subsubsection}}\Ssubsubsectionstarx{#1}{#2}\setcounter{subsubsection}{\value{GrouperTemp}}}
\newcommand{\Ssubsubsubsectiongrouperstarx}[2]{\Ssubsubsubsectionstarx{#1}{#2}}
\newcommand{\Ssubsubsubsubsectiongrouperstarx}[2]{\Ssubsubsubsubsectionstarx{#1}{#2}}

% Generated by `subsubsub*section':
\newcommand{\SSubSubSubSection}[1]{\Ssubsubsubsubsectionstar{#1}}

% For hidden parts with an empty title:
\newcommand{\notitlesection}{\vspace{2ex}\phantomsection\noindent}

% To increment section numbers:
\newcommand{\Sincpart}{\stepcounter{part}}
\newcommand{\Sincsection}{\stepcounter{section}}
\newcommand{\Sincsubsection}{\stepcounter{subsection}}
\newcommand{\Sincsubsubsection}{\stepcounter{subsubsection}}
\newcommand{\Sincsubsubsubsection}{}
\newcommand{\Sincsubsubsubsubsection}{}

% When brackets appear in section titles:
\newcommand{\SOpenSq}{[}
\newcommand{\SCloseSq}{]}

% Helper for box-mode macros:
\newcommand{\Svcenter}[1]{$\vcenter{#1}$}

% Verbatim
\newenvironment{SVerbatim}{}{}

% Helper to work around a problem with "#"s for URLs within \href
% within other macros:
\newcommand{\Shref}[3]{\href{#1\##2}{#3}}

% For URLs:
\newcommand{\Snolinkurl}[1]{\nolinkurl{#1}}

% History note:
\newcommand{\SHistory}[1]{\begin{smaller}#1\end{smaller}}

%%%%%%%%%%%%%%%%%%%%%%%%%%%%%%%%%%%%%%%%%%%%%%%%%%%%%%%%%%%%%%%%%%%%%%%%%%%%%%%%

% Scribble then generates the following:
%
%  \begin{document}
%  \preDoc
%  \titleAndVersion{...}{...}
%  ... document content ...
%  \postDoc
%  \end{document}
%%%%%%%%%%%%%%%%%%%%%%%%%%%%%%%%%%%%%%%%%%%%%%%%%%%%%%%%%%%%%%%%%%%%%%%%%%%%%%%%
% BEGIN acmart/acmart.tex
% Support for styles in scribble/acmart

% These are replaced by scribble/acmart/style.tex,
%  which is used in combination with acmart.cls

\newcommand{\SAuthorinfo}[4]{#1}
\newcommand{\SAuthorPlace}[1]{#1}
\newcommand{\SAuthorEmail}[1]{#1}
\newcommand{\SAuthorOrcid}[1]{#1}

\newcommand{\SConferenceInfo}[2]{}
\newcommand{\SCopyrightYear}[1]{}
\newcommand{\SCopyrightData}[1]{}
\newcommand{\Sdoi}[1]{}
\newcommand{\SPexclusivelicense}[0]{}

\newcommand{\SCategory}[3]{}
\newcommand{\SCategoryPlus}[4]{}
\newcommand{\STerms}[1]{}
\newcommand{\SKeywords}[1]{}

% Normally gets re-written by the title macro:
\newcommand{\SSubtitle}[1]{{\bf #1}}

% Use ACM color; it would be better to use `citecolor` here somehow,
% but I can't figure out how to do that
\newcommand{\AutobibLink}[1]{\color{ACMPurple}{#1}}
% END acmart/acmart.tex
%%%%%%%%%%%%%%%%%%%%%%%%%%%%%%%%%%%%%%%%%%%%%%%%%%%%%%%%%%%%%%%%%%%%%%%%%%%%%%%%

% Redefine \SColorize to produce B&W Scheme text
\newcommand{\SColorize}[2]{\color{#1}{#2}}

% Redefine SHyphen to allow identifiers to be hyphenated
\newcommand{\SHyphen}[1]{#1}

\newcommand{\inColor}[2]{{\SHyphen{\Scribtexttt{\SColorize{#1}{#2}}}}}
\definecolor{PaleBlue}{rgb}{0.90,0.90,1.0}
\definecolor{LightGray}{rgb}{0.90,0.90,0.90}
\definecolor{CommentColor}{rgb}{0.76,0.45,0.12}
\definecolor{ParenColor}{rgb}{0.52,0.24,0.14}
\definecolor{IdentifierColor}{rgb}{0.15,0.15,0.50}
\definecolor{ResultColor}{rgb}{0.0,0.0,0.69}
\definecolor{ValueColor}{rgb}{0.13,0.55,0.13}
\definecolor{OutputColor}{rgb}{0.59,0.00,0.59}

\newcommand{\RktPlain}[1]{\inColor{black}{#1}}
\newcommand{\RktKw}[1]{{\SColorize{black}{\Scribtexttt{#1}}}} % no \textbf anymore
\newcommand{\RktStxLink}[1]{\RktKw{#1}}
\newcommand{\RktStxDef}[1]{\RktStxLink{#1}}
\newcommand{\RktCmt}[1]{\inColor{CommentColor}{#1}}
\newcommand{\RktPn}[1]{\inColor{ParenColor}{#1}}
\newcommand{\RktInBG}[1]{\inColor{ParenColor}{#1}}
\newcommand{\RktSym}[1]{\inColor{IdentifierColor}{#1}}
\newcommand{\RktSymDef}[1]{\RktSym{#1}}
\newcommand{\RktVal}[1]{\inColor{ValueColor}{#1}}
\newcommand{\RktValLink}[1]{\inColor{blue}{#1}}
\newcommand{\RktValDef}[1]{\RktValLink{#1}}
\newcommand{\RktModLink}[1]{\inColor{blue}{#1}}
\newcommand{\RktRes}[1]{\inColor{ResultColor}{#1}}
\newcommand{\RktOut}[1]{\inColor{OutputColor}{#1}}
\newcommand{\RktMeta}[1]{\inColor{IdentifierColor}{#1}}
\newcommand{\RktMod}[1]{\inColor{black}{#1}}
\newcommand{\RktRdr}[1]{\inColor{black}{#1}}
\newcommand{\RktVarCol}[1]{\inColor{IdentifierColor}{#1}}
\newcommand{\RktVar}[1]{{\RktVarCol{\textsl{#1}}}}
\newcommand{\RktErrCol}[1]{\inColor{red}{#1}}
\newcommand{\RktErr}[1]{{\RktErrCol{\textrm{\textit{#1}}}}}
\newcommand{\RktOpt}[1]{#1}
\newcommand{\RktIn}[1]{\incolorbox{LightGray}{\RktInBG{#1}}}
\newcommand{\highlighted}[1]{\colorbox{PaleBlue}{\hspace{-0.5ex}\RktInBG{#1}\hspace{-0.5ex}}}

\newenvironment{RktBlk}{}{}
\newenvironment{defmodule}{}{}
\newenvironment{prototype}{}{}
\newenvironment{argcontract}{}{}
\newenvironment{together}{}{}

\newenvironment{specgrammar}{}{}


\newenvironment{RBibliography}{}{}
\newcommand{\bibentry}[1]{\parbox[t]{0.8\linewidth}{#1}}

\newenvironment{leftindent}{\begin{quote}}{\end{quote}}
\newenvironment{insetpara}{\begin{quote}}{\end{quote}}

\newcommand{\Rfilebox}[2]{\begin{list}{}{\topsep=0pt\partopsep=0pt%
\listparindent=0pt\itemindent=0pt\labelwidth=0pt\leftmargin=0ex\rightmargin=0ex%
\itemsep=0pt\parsep=0pt}\item #1

#2\end{list}}
\newcommand{\RfileboxBox}[3]{#3{\halign{\hfil##\cr #1 \cr #2 \cr}}}
\newcommand{\RfileboxBoxT}[2]{\RfileboxBox{#1}{#2}{\vtop}}
\newcommand{\RfileboxBoxC}[2]{\RfileboxBox{#1}{#2}{\Svcenter}}
\newcommand{\RfileboxBoxB}[2]{\RfileboxBox{#1}{#2}{\vbox}}
\newcommand{\Rfiletitle}[1]{\hfill \fbox{#1}}
\newcommand{\RfiletitleBox}[1]{\fbox{#1}}
\newcommand{\Rfilename}[1]{#1}
\newenvironment{Rfilecontent}{}{}
\newcommand{\RfilecontentBox}[1]{#1}

\newcommand{\RBackgroundLabel}[1]{}
\newenvironment{RBackgroundLabelInner}{}{}

\newcommand{\RpackageSpec}[1]{\hspace{5ex} #1}
%%%%%%%%%%%%%%%%%%%%%%%%%%%%%%%%%%%%%%%%%%%%%%%%%%%%%%%%%%%%%%%%%%%%%%%%%%%%%%%%
% BEGIN acmart/style.tex

\renewcommand{\titleAndVersionAndAuthors}[3]{\title{#1}#3\maketitle}
\renewcommand{\titleAndEmptyVersionAndAuthors}[3]{\titleAndVersionAndAuthors{#1}{#2}{#3}}
\renewcommand{\titleAndVersionAndEmptyAuthors}[3]{\title{#1}\author{Anonymous Author(s)}\maketitle}
\renewcommand{\titleAndEmptyVersionAndEmptyAuthors}[3]{\titleAndVersionAndEmptyAuthors{#1}{#2}{#3}}
\renewcommand{\titleAndVersionAndAuthorsAndShort}[4]{\title[#4]{#1}#3\maketitle}
\renewcommand{\titleAndEmptyVersionAndAuthorsAndShort}[4]{\titleAndVersionAndAuthorsAndShort{#1}{#2}{#3}{#4}}
\renewcommand{\titleAndVersionAndEmptyAuthorsAndShort}[4]{\title[#4]{#1}\author{Anonymous Author(s)}\maketitle}
\renewcommand{\titleAndEmptyVersionAndEmptyAuthorsAndShort}[4]{\titleAndVersionAndEmptyAuthorsAndShort{#1}{#2}{#3}{#4}}

% Support plain `author' while enabling `authorinfo': for each
% use of \SAuthor, check whether it contains an \SAuthorinfo form:
\def\SAuthor#1{\SAutoAuthor#1\SAutoAuthorDone{#1}}
\def\SAutoAuthorDone#1{}
\def\SAutoAuthor{\futurelet\next\SAutoAuthorX}
\def\SAutoAuthorX{\ifx\next\SAuthorinfo \let\Snext\relax \else \let\Snext\SToAuthorDone \fi \Snext}
\def\SToAuthorDone{\futurelet\next\SToAuthorDoneX}
\def\SToAuthorDoneX#1{\ifx\next\SAutoAuthorDone \let\Snext\SAddAuthorInfo \else \let\Snext\SToAuthorDone \fi \Snext}
\newcommand{\SAddAuthorInfo}[1]{\SAuthorinfo{#1}{}{}}

\renewcommand{\SAuthorinfo}[4]{\author{#1}{#2}{#3}{#4}}
\renewcommand{\SAuthorSep}[1]{}
\renewcommand{\SAuthorOrcid}[1]{\orcid{#1}}
\renewcommand{\SAuthorPlace}[1]{\affiliation{#1}}
\renewcommand{\SAuthorEmail}[1]{\email{#1}}

\renewcommand{\SConferenceInfo}[2]{\conferenceinfo{#1}{#2}}
\renewcommand{\SCopyrightYear}[1]{\copyrightyear{#1}}
\renewcommand{\SCopyrightData}[1]{\copyrightdata{#1}}
%\renewcommand{\Sdoi}[1]{\doi{#1}}
\renewcommand{\SPexclusivelicense}[0]{\exclusivelicense}

\renewcommand{\SCategory}[3]{\category{#1}{#2}{#3}}
\renewcommand{\SCategoryPlus}[4]{\category{#1}{#2}{#3}[#4]}
\renewcommand{\STerms}[1]{\terms{#1}}
\renewcommand{\SKeywords}[1]{\keywords{#1}}

\newcommand{\StitleShort}[2]{\title[#1]{#2}}

\newcommand{\Sshortauthors}[1]{\renewcommand{\shortauthors}{#1}}

\newcommand{\SacmBadgeRURL}[2]{\acmBadgeR[#1]{#2}}
\newcommand{\SacmBadgeLURL}[2]{\acmBadgeL[#1]{#2}}

\newcommand{\SgrantnumURL}[3]{\grantnum[#1]{#2}{#3}}

\newcommand{\SreceivedStage}[2]{\received[#1]{#2}}

\newcommand{\SccsdescNumber}[2]{\ccsdesc[#1]{#2}}
% END acmart/style.tex
%%%%%%%%%%%%%%%%%%%%%%%%%%%%%%%%%%%%%%%%%%%%%%%%%%%%%%%%%%%%%%%%%%%%%%%%%%%%%%%%
\begin{document}
\preDoc


\begin{abstract}Music composition is an artistic process in which a composer analyzes musical
language from a diverse array of cultures in order to synthesize new works. The
conscientious composer spends significant time experimenting in multi{-}language
music ecosystems and translating the results of these experiments into
performable or listenable output.  This demo introduces Tonart, an extensible
system for composing and compiling works expressed in multiple music languages.
Terms from the chosen languages are embedded in a "score", a coordinate system
abstraction over a physical musical score.  The score is then given semantics
in terms of a known music language, using a specified sequence of rewrites.
The score is accessible at all points in the sequence.  Tonart{'}s rewriters
leverage context{-}sensitivity and locality, corresponding to how notations
interact when written on physical scores.  Crucially, it is the composer{'}s
choice how conflicts in different languages{'} usual semantics are addressed.
Moreover, using local rewrites, sections of the score may be given
independent semantics.  In this way, Tonart compiles source multi{-}language
scores into performable target scores, which may be further compiled into
computer performances or human{-}readable sheet music.\end{abstract}\titleAndEmptyVersionAndAuthors{Demo: Composable Compositions with Tonart}{}{\SNumberOfAuthors{1}\SAuthor{\SAuthorinfo{Jared Gentner}{}{\SAuthorPlace{}}{\SAuthorEmail{jagen315@gmail.com}}}}
\label{t:x28part_x22Demox5fx5fComposablex5fCompositionsx5fwithx5fTonartx22x29}





\Sshortauthors{J. Gentner}

\sectionNewpage

\Ssection{Context}{Context}\label{t:x28part_x22Contextx22x29}

In this demo, we will compose a piece of music borrowing language from several different
musical and non{-}musical traditions.  Some of the concepts are unknown to
musicians, so it{'}s not an expectation that the viewer is familiar with them,
either.  However, we will compile to commonly known music objects, which are
summarized below [and will be re{-}explained, and visualized, in the live
session].

\Ssubsection{Tones}{Tones}\label{t:x28part_x22Tonesx22x29}

For our purposes, a tone is a sound which is produced by a sine wave
at a given frequency.

\begin{SCodeFlow}\begin{RktBlk}\begin{SingleColumn}\RktSym{{\Stttextless}freq{\Stttextmore}Hz}\RktMeta{}\mbox{\hphantom{\Scribtexttt{x}}}\RktMeta{}\RktCmt{;}\mbox{\hphantom{\Scribtexttt{x}}}\RktCmt{Standard}\mbox{\hphantom{\Scribtexttt{x}}}\RktCmt{Syntax}\RktMeta{}

\RktMeta{}\RktSym{440Hz}\RktMeta{}\mbox{\hphantom{\Scribtexttt{x}}}\RktMeta{}\RktCmt{;}\mbox{\hphantom{\Scribtexttt{x}}}\RktCmt{Example}\RktMeta{}

\RktMeta{~}

\RktMeta{}\RktPn{(}\RktSym{tone}\RktMeta{}\mbox{\hphantom{\Scribtexttt{x}}}\RktMeta{}\RktSym{{\Stttextless}freq{\Stttextmore}}\RktPn{)}\RktMeta{}\mbox{\hphantom{\Scribtexttt{x}}}\RktMeta{}\RktCmt{;}\mbox{\hphantom{\Scribtexttt{x}}}\RktCmt{Tonart}\mbox{\hphantom{\Scribtexttt{x}}}\RktCmt{Syntax}\RktMeta{}

\RktMeta{}\RktPn{(}\RktSym{tone}\RktMeta{}\mbox{\hphantom{\Scribtexttt{x}}}\RktMeta{}\RktVal{440}\RktPn{)}\RktMeta{}\mbox{\hphantom{\Scribtexttt{xx}}}\RktMeta{}\RktCmt{;}\mbox{\hphantom{\Scribtexttt{x}}}\RktCmt{Example}\RktMeta{}\end{SingleColumn}\end{RktBlk}\end{SCodeFlow}

A special property of how we perceive
tones is: two tones that are double each other{'}s frequency sound "almost the
same" (are hard to tell apart).  This is why we can all sing the same songs
together even though we have different vocal ranges.

\Ssubsection{Notes}{Notes}\label{t:x28part_x22Notesx22x29}

For our purposes, notes are a standard meter laid over top of tones
(frequencies).

\begin{SCodeFlow}\begin{RktBlk}\begin{SingleColumn}\RktCmt{;;}\mbox{\hphantom{\Scribtexttt{x}}}\RktCmt{Standard}\mbox{\hphantom{\Scribtexttt{x}}}\RktCmt{Syntax}\RktMeta{}

\RktMeta{}\RktSym{{\Stttextless}pitch{-}label{\Stttextmore}{\Stttextless}accidental{\Stttextmore}{\Stttextless}octave{\Stttextmore}}\RktMeta{}

\RktMeta{}\RktSym{Ab5}\RktMeta{}\mbox{\hphantom{\Scribtexttt{x}}}\RktMeta{}\RktCmt{;}\mbox{\hphantom{\Scribtexttt{x}}}\RktCmt{Example}\RktMeta{}

\RktMeta{~}

\RktMeta{}\RktCmt{;;}\mbox{\hphantom{\Scribtexttt{x}}}\RktCmt{Tonart}\mbox{\hphantom{\Scribtexttt{x}}}\RktCmt{Syntax}\RktMeta{}

\RktMeta{}\RktPn{(}\RktSym{note}\RktMeta{}\mbox{\hphantom{\Scribtexttt{x}}}\RktMeta{}\RktSym{{\Stttextless}pitch{-}label{\Stttextmore}}\RktMeta{}\mbox{\hphantom{\Scribtexttt{x}}}\RktMeta{}\RktSym{{\Stttextless}accidental{-}number{\Stttextmore}}\RktMeta{}\mbox{\hphantom{\Scribtexttt{x}}}\RktMeta{}\RktSym{{\Stttextless}octave{-}number{\Stttextmore}}\RktPn{)}\RktMeta{}

\RktMeta{}\RktPn{(}\RktSym{note}\RktMeta{}\mbox{\hphantom{\Scribtexttt{x}}}\RktMeta{}\RktSym{a}\RktMeta{}\mbox{\hphantom{\Scribtexttt{x}}}\RktMeta{}\RktVal{{-}1}\RktMeta{}\mbox{\hphantom{\Scribtexttt{x}}}\RktMeta{}\RktVal{5}\RktPn{)}\RktMeta{}\mbox{\hphantom{\Scribtexttt{x}}}\RktMeta{}\RktCmt{;}\mbox{\hphantom{\Scribtexttt{x}}}\RktCmt{Example}\RktMeta{}\end{SingleColumn}\end{RktBlk}\end{SCodeFlow}

Notes are given labels A1 B1 C1 D1 E1 F1 G1 A2 B2 C2 D2 ...,
which, in that order, must have increasing corresponding frequencies.  C2 must
have a frequency exactly double C1.  We call the distance between two notes an
"interval" and this particular same{-}sounding, doubled{-}frequency interval an
"octave".  Two notes an octave apart will always have the same $\_$pitch{-}label$\_$,
just as in the above example.  The octave number is what will vary (C1, C2, C3,
C4 etc.).  An expanded set of tones can be expressed as a raising or lowering of
a note, which is done via an "accidental".  To raise a note, the symbol {'}\#{'}
(pronounced "sharp") is used.  To lower a note, the symbol {'}b{'} (pronounced
{'}flat{'}) is used. The accidental is also preserved between octaves, and the
pitch{-}label taken with the octave is called the "pitch class".  For example, a
raised C is written C\#, each distinct pitch classes.  C\# in the 4th octave is
written C\#4.  The C\# an octave above C\#4 is written C\#5.  For clairty{-} C\#4 and
C\#5 are in the same pitch class and that pitch class is C\#.

\Ssubsection{Tuning}{Tuning}\label{t:x28part_x22Tuningx22x29}

A tuning is a named precise mapping of notes to tones.  This usually means mapping
one octave of symbols{-} C \#/b D \#/b E F \#/b G \#/b A \#/b B to reasonable frequencies,
and deriving all the others by doubling or halving those frequencies
(transposing by octaves).

\begin{SCodeFlow}\begin{RktBlk}\begin{SingleColumn}\RktCmt{;;}\mbox{\hphantom{\Scribtexttt{x}}}\RktCmt{Standard}\mbox{\hphantom{\Scribtexttt{x}}}\RktCmt{Syntax}\RktMeta{}

\RktMeta{}\RktSym{{\Stttextless}tuning{-}name{\Stttextmore}{\Stttextless}note{\Stttextmore}={\Stttextless}frequency{\Stttextmore}}\RktMeta{}

\RktMeta{}\RktCmt{;;}\mbox{\hphantom{\Scribtexttt{x}}}\RktCmt{Example}\RktMeta{}

\RktMeta{}\RktSym{12tet}\RktMeta{}\mbox{\hphantom{\Scribtexttt{x}}}\RktMeta{}\RktSym{a=440Hz}\RktMeta{}

\RktMeta{~}

\RktMeta{}\RktCmt{;;}\mbox{\hphantom{\Scribtexttt{x}}}\RktCmt{Tonart}\mbox{\hphantom{\Scribtexttt{x}}}\RktCmt{Syntax}\RktMeta{}

\RktMeta{}\RktPn{(}\RktSym{tuning}\RktMeta{}\mbox{\hphantom{\Scribtexttt{x}}}\RktMeta{}\RktSym{{\Stttextless}tuning{-}name{\Stttextmore}}\RktMeta{}\mbox{\hphantom{\Scribtexttt{x}}}\RktMeta{}\RktPn{[}\RktSym{{\Stttextless}note{\Stttextmore}}\RktMeta{}\mbox{\hphantom{\Scribtexttt{x}}}\RktMeta{}\RktSym{{\Stttextless}frequency{-}number{\Stttextmore}}\RktPn{]}\RktPn{)}\RktMeta{}

\RktMeta{}\RktCmt{;;}\mbox{\hphantom{\Scribtexttt{x}}}\RktCmt{Example}\RktMeta{}

\RktMeta{}\RktPn{(}\RktSym{tuning}\RktMeta{}\mbox{\hphantom{\Scribtexttt{x}}}\RktMeta{}\RktSym{12tet}\RktMeta{}\mbox{\hphantom{\Scribtexttt{x}}}\RktMeta{}\RktPn{[}\RktSym{a}\RktMeta{}\mbox{\hphantom{\Scribtexttt{x}}}\RktMeta{}\RktVal{440}\RktPn{]}\RktPn{)}\RktMeta{}\end{SingleColumn}\end{RktBlk}\end{SCodeFlow}

Many tuning systems exist.   When browsing tuning
systems, recognize that these are typically uniform ways to tune a keyboard,
with compromises based around what kind of music is to be played.  Moreover, a
performance practice for small groups or vocalists (without keyboard
accompaniment) will de facto employ a "context sensitive tuning", as the ear and
instrument are capable of retuning notes to be closer to "more correct"
intervals.  I will not speculate on what "more correct" means here, but I can
safely say that static tuning systems are forced to approximate intended
intervals in many situations.  And that tuning on the whole is not a matter of
some kind of mathematical correctness, it is rather a matter of enhancing
expression.

\Ssubsection{Chords}{Chords}\label{t:x28part_x22Chordsx22x29}

Chords are an abstraction over combinations of notes, which often function like
a set of pitch classes.  For the purposes of this demo, our chord symbols will
specify sets of 3 pitch classes, called triads.

\begin{SCodeFlow}\begin{RktBlk}\begin{SingleColumn}\RktCmt{;;}\mbox{\hphantom{\Scribtexttt{x}}}\RktCmt{Standard}\mbox{\hphantom{\Scribtexttt{x}}}\RktCmt{Syntax}\RktMeta{}

\RktMeta{}\RktSym{{\Stttextless}pitch{-}label{\Stttextmore}{\Stttextless}accidental{\Stttextmore}{\Stttextless}quality{-}abbrev{\Stttextmore}}\RktMeta{}\mbox{\hphantom{\Scribtexttt{x}}}\RktMeta{}\RktSym{or}\RktMeta{}\mbox{\hphantom{\Scribtexttt{x}}}\RktMeta{}\RktSym{{\Stttextless}pitch{-}class{\Stttextmore}}\RktMeta{}\mbox{\hphantom{\Scribtexttt{x}}}\RktMeta{}\RktSym{{\Stttextless}quality{\Stttextmore}}\RktMeta{}

\RktMeta{}\RktCmt{;;}\mbox{\hphantom{\Scribtexttt{x}}}\RktCmt{Example}\RktMeta{}

\RktMeta{}\RktSym{F\#m}\RktMeta{}\mbox{\hphantom{\Scribtexttt{x}}}\RktMeta{}\RktSym{or}\RktMeta{}\mbox{\hphantom{\Scribtexttt{x}}}\RktMeta{}\RktSym{F\#}\RktMeta{}\mbox{\hphantom{\Scribtexttt{x}}}\RktMeta{}\RktSym{minor}\RktMeta{}

\RktMeta{~}

\RktMeta{}\RktCmt{;;}\mbox{\hphantom{\Scribtexttt{x}}}\RktCmt{Tonart}\mbox{\hphantom{\Scribtexttt{x}}}\RktCmt{Syntax}\RktMeta{}

\RktMeta{}\RktPn{(}\RktSym{chord}\RktMeta{}\mbox{\hphantom{\Scribtexttt{x}}}\RktMeta{}\RktSym{{\Stttextless}pitch{-}label{\Stttextmore}}\RktMeta{}\mbox{\hphantom{\Scribtexttt{x}}}\RktMeta{}\RktSym{{\Stttextless}accidental{-}number{\Stttextmore}}\RktMeta{}\mbox{\hphantom{\Scribtexttt{x}}}\RktMeta{}\RktPn{[}\RktSym{{\Stttextless}quality{-}abbrev{\Stttextmore}}\RktPn{]}\RktPn{)}\RktMeta{}

\RktMeta{}\RktCmt{;;}\mbox{\hphantom{\Scribtexttt{x}}}\RktCmt{Example}\RktMeta{}

\RktMeta{}\RktPn{(}\RktSym{chord}\RktMeta{}\mbox{\hphantom{\Scribtexttt{x}}}\RktMeta{}\RktSym{f}\RktMeta{}\mbox{\hphantom{\Scribtexttt{x}}}\RktMeta{}\RktVal{1}\RktMeta{}\mbox{\hphantom{\Scribtexttt{x}}}\RktMeta{}\RktPn{[}\RktSym{m}\RktPn{]}\RktPn{)}\RktMeta{}\end{SingleColumn}\end{RktBlk}\end{SCodeFlow}

While chords in some music are just an analytical or summary tool with
debatable real existance, for many guitarists chord symbols and strum patterns
are the primary building blocks of their musical language.  Jazz charts are both
summarized in chords and written in chords, though jazz chords typically have 4,
5, 6 or more members in their corresponding pitch class sets, to phrase it in
our established terminology.

Any number of notes can be in a chord, including repeats and octave transpositions.
Not all pitch classes from the set need be present for a chord to retain its identity, either.
It is important to remember that these are abstractions created to describe the
phenomena we experience.  For example, a pipe organ playing a single notated
note may actually be playing a chord of some variety.  Any instrument sounds a
whole series of tones which we consider to be dominated by the lowest one,
called the fundamental frequency. It is possible to expand the score into one
which reflects some musical reality based on the notes, the instruments, the
chords, the room, the weather that day etc. and then analyze that, but we are
always going to be applying thresholds to determine what is significant.  Anyway
having a toolkit to facilitate this kind of thing is useful.

\postDoc
\end{document}
